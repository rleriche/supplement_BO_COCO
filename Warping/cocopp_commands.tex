\providecommand{\bbobecdfcaptionsinglefunctionssingledim}[1]{
Empirical cumulative distribution of simulated (bootstrapped)
             runtimes, measured in number of objective function evaluations 
             divided by dimension (FEvals/DIM) in 
             dimension #1 and for those targets in
             $10^{[-8..2]}$ that have just not been reached by 
             the reference algorithm in a given budget of $k$ $\times$ DIM, with 
             $8$ different values of $k$ chosen 
             equidistant in logscale within the interval $\{0.5, \dots, 50\}$.
}
\providecommand{\cocoversion}{\hspace{\textwidth}\scriptsize\sffamily{}\color{Gray}Data produced with COCO v2.3.3.84}
\providecommand{\numofalgs}{7}
\providecommand{\bbobECDFslegend}[1]{
Bootstrapped empirical cumulative distribution of the number of objective function evaluations divided by dimension (FEvals/DIM) for all functions and subgroups in #1-D. The targets are chosen from $10^{[-8..2]}$ such that the reference algorithm just not reached them within a given budget of $k$ $\times$ DIM, with $8$ different values of $k$ chosen equidistant in logscale within the interval $\{0.5, \dots, 50\}$. As reference algorithm, the reference algorithm is shown as light thick line with diamond markers.
}
\providecommand{\bbobppfigslegend}[1]{
Expected running time (\ERT\ in number of $f$-evaluations
                        as $\log_{10}$ value) divided by dimension versus dimension. The target function value
                        is chosen such that the reference algorithm just failed to achieve
                        an \ERT\ of $10\times\DIM$. Different symbols correspond to different algorithms given in the legend of #1. Light symbols give the maximum number of function evaluations from the longest trial divided by dimension. Black stars indicate a statistically better result compared to all other algorithms with $p<0.01$ and Bonferroni correction number of dimensions (six).  
Legend: 
{\color{NavyBlue}$\circ$}: \algorithmA
, {\color{Magenta}$\diamondsuit$}: \algorithmB
, {\color{Orange}$\star$}: \algorithmC
, {\color{CornflowerBlue}$\triangledown$}: \algorithmD
, {\color{red}$\varhexagon$}: \algorithmE
, {\color{YellowGreen}$\triangle$}: \algorithmF
, {\color{cyan}$\pentagon$}: \algorithmG
}
% define some COCO/dvipsnames colors because
% ACM style does not allow to use them directly
\definecolor{NavyBlue}{HTML}{000080}
\definecolor{Magenta}{HTML}{FF00FF}
\definecolor{Orange}{HTML}{FFA500}
\definecolor{CornflowerBlue}{HTML}{6495ED}
\definecolor{YellowGreen}{HTML}{9ACD32}
\definecolor{Gray}{HTML}{BEBEBE}
\definecolor{Yellow}{HTML}{FFFF00}
\definecolor{GreenYellow}{HTML}{ADFF2F}
\definecolor{ForestGreen}{HTML}{228B22}
\definecolor{Lavender}{HTML}{FFC0CB}
\definecolor{SkyBlue}{HTML}{87CEEB}
\definecolor{NavyBlue}{HTML}{000080}
\definecolor{Goldenrod}{HTML}{DDF700}
\definecolor{VioletRed}{HTML}{D02090}
\definecolor{CornflowerBlue}{HTML}{6495ED}
\definecolor{LimeGreen}{HTML}{32CD32}

\providecommand{\ntables}{8}
\providecommand{\pptablesfooter}{
\end{tabularx}
}
\providecommand{\pptablesheader}{
\begin{tabularx}{1.0\textwidth}{@{}c@{}|*{8}{@{}r@{}X@{}}|@{}r@{}@{}l@{}}
\#FEs/D & \multicolumn{2}{@{}c@{}}{0.5} & \multicolumn{2}{@{}c@{}}{1} & \multicolumn{2}{@{}c@{}}{3} & \multicolumn{2}{@{}c@{}}{5} & \multicolumn{2}{@{}c@{}}{7} & \multicolumn{2}{@{}c@{}}{10} & \multicolumn{2}{@{}c@{}}{15} & \multicolumn{2}{@{}c@{}}{20} & \multicolumn{2}{|@{}l@{}}{\#succ}\\\hline
}
\providecommand{\algGtables}{\StrLeft{WarpL}{\ntables}}
\providecommand{\algFtables}{\StrLeft{WarpS}{\ntables}}
\providecommand{\algEtables}{\StrLeft{WarpM}{\ntables}}
\providecommand{\algDtables}{\StrLeft{L}{\ntables}}
\providecommand{\algCtables}{\StrLeft{S}{\ntables}}
\providecommand{\algBtables}{\StrLeft{M}{\ntables}}
\providecommand{\algAtables}{\StrLeft{random}{\ntables}}
\providecommand{\ntables}{8}
\providecommand{\ntables}{8}
\providecommand{\ntables}{8}
\providecommand{\bbobpptablesmanylegend}[1]{%
        Expected runtime (\ERT\ in number of function 
        evaluations) divided by the respective the \ERT\ of the reference algorithm in
        #1.
        This \ERT\ ratio and, in braces as dispersion measure, the half difference between
        10 and 90\%-tile of bootstrapped run lengths appear for each algorithm and 
        %
        run-length based target, the corresponding reference \ERT\
        (preceded by the target \Df-value in \textit{italics}) in the first row. 
        \#succ is the number of trials that reached the target value of the last column.
        %
        The median number of conducted function evaluations is additionally given in 
        \textit{italics}, if the target in the last column was never reached.
        Entries, succeeded by a star, are statistically significantly better (according to
        the rank-sum test) when compared to all other algorithms of the table, with
        $p = 0.05$ or $p = 10^{-k}$ when the number $k$ following the star is larger
        than 1, with Bonferroni correction by the number of functions (24). A $\downarrow$ indicates the same tested against the reference algorithm. Best results are printed in bold.
        \cocoversion}
\providecommand{\algsfolder}{rando_M_S_L_WarpM_WarpS_WarpL_et_al/}
\providecommand{\algorithmA}{L}
\providecommand{\algorithmB}{M}
\providecommand{\algorithmC}{S}
\providecommand{\algorithmD}{WarpL}
\providecommand{\algorithmE}{WarpM}
\providecommand{\algorithmF}{WarpS}
\providecommand{\algorithmG}{random}
